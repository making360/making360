
%----------------------------------------------------------------------------------------
%	INTRODUCTION
%----------------------------------------------------------------------------------------
\SkipTocEntry\section*{introduction}
\addtocounter{section}{2}
%------------------------------------------------
\begin{fullwidth}

{\itshape\bfseries написано Джейсоном Флетчером/Jason Fletcher

}

Кто бы что ни говорил, а съемка видео 360 - это сложный процесс! Это медиум с абсолютно уникальными задачами и проблемами. Перед тем как добиться успеха, нужно будет преодолеть массу сложностей.  Знание всех особенностей, ограничений и потенциальных пробрем поможет успешно создать эффект погружения. Именно об это и пойдт речь в этой книге.

Мы собираемся сделать выброс информации. И все же, соединение ее воедино и понимание оптимального процесса работы для твоей системы зависит от тебя. Это не типичная "Как сделать самому" книга. Чтобы дейтвительно стать адептом видео 360, вам придется делать тестовые съемки и вникать в проблемы самому. Лучший способ научиться и получить ценный опыт - это провал! Тем не менее, мы предоставим на вооружение всесторонный подход.

{\large Big Picture Workflow\par}

Многие аспекты требуют объяснений. И так мы говорим о подходе с применением грубой силы, собранной в этих главах. Но в реальности существуют определенные шаги в стандартной панарамной видео съемке:

\begin{itemize}
\item Оборудование: Выбор системы
\item Установка: настройки камеры, карт памяти, пульта
\item План: стаблизация, зоны безопасности, блокировка
\item Съемка: запись, синхронизация, свет
\item Импор: обработка, управление файлами 
\item Сшивка: согласование цветов, рендер tiff
\item Редактирование: rotoscoping, цветокоррекция, итоговый рендер
\end{itemize}

\clearpage
\end{fullwidth}
%----------------------------------------------------------------------------------------
