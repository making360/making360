\chapter{Lighting}
\pagecolor{white}
\label{chap:25}
\begin{fullwidth}


\problem

{\large You want the most optimal lighting conditions for the shoot. \par}

Lighting is tricky with the GoPros. With low lighting conditions, the image has a lot of noise. Too much artificial lighting causes a variety of problems such as a blown out image, pollution, and color variation. Lens flares are also more common with wide angle and fisheye lens. If shooting in stereo, the flares and differences cause a jarring image. Also, where do you hide the lights??

\solution

{\large Stay natural. 
 \par}

When shooting outdoors, if you have the time and patience, wait for the right moment. Try to shoot after before dusk when the sun has just set but still emits a light hue. However, there is a small window to catch the perfect timing. If shooting at a different time of day, the sun will be pointing directly into one of the cameras, causing overexposure. You can shoot the moon and patch the nadir if it is the camera pointing towards the sky. 

For interior shots, do not use too many different artificial lights. They will cause various colors and shadows. Unless you are shooting a JJ Abrams style VR piece, be careful with pointing light sources directly and the lens, creating lens flares.

Be careful not to use too much tungsten lighting or the infrared pollution will cause a purplish hue and need to be color corrected. 

Everything will show in a 360 shot and unfortunately you cannot hide and seek with lights. Try dressing your set and hiding light sources in blind spots. 

NEED CONTRIBUTORS

\clearpage
\end{fullwidth}