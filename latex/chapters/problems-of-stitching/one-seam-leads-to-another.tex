\section{One Seam Leads to Another}
\pagecolor{white}
\label{chap:38}
\begin{fullwidth}
\group{stitch}

{\itshape\bfseries “Every living being is an engine geared to the wheelwork of the universe. Though seemingly affected only by its immediate surrounding, the sphere of external influence extends to infinite distance.”}

- Nikola Tesla, How Cosmic Forces Shape Our Destinies
\vspace{\baselineskip}

\problem

{\large As you are fixing a seam, one or two other seams start to appear. \par}

You only have one small seam to fix before rendering. You make some quick changes, almost done! You preview just a few seconds and one or two seams just showed up out of nowhere. Should you have even fixed the small seam in the first place or should you have previewed the entire video before fixing any seams?

\solutions

{\large Gotta get them all at once! \par}

To quickly fix all the seams, find most of them in one frame. 

\imgB{.5}{38/oneseam}{38/4seams}
\clearpage
The best playback for testing is Quicktime or VLC. AVP is great for previewing but not optimal for real time playback at the actual FPS. This will cause you to miss some seams. Take notes of the frames where seams require some work while viewing your \textbf{\nameref{chap:30}}.

\imgA{1}{38/rangestitch}

When you are ready to fine stitch, reopen your previous kava project or start a new one by dragging your videos into AVP. Select the frame with the most amount of seams and start by fixing all the seams in the frame. Update AVP by saving the template in APG.

Using the blue range selector in AVP will improve the average quality of the stitch for the selected range, based on your in and out points. The Optimizer engine and stitching algorithm will focus on that range, instead of the beginning where your DP’s face is all over each camera, unstitchable!

\imgA{1}{38/range}

Fixing all the seams at once makes it easier to prevent new seams to show up because you already have an overview of all the worst seams. Make a plan of attack that conquers all the large seams at once. Then the small seams can be fixed with \textbf{\nameref{chap:41}} or a simple optimization.

\imgA{1}{38/fixonce}

{\large Focus on the creative! \par}

Most seams are difficult to fix when there are subjects or moving objects in the scene. Plan and storyboard ahead. Lack of pre-production effort causes many seams and issues in post. Choosing the wrong rig, not carefully planning the movements of your subjects, handling lighting like a traditional film production, barely rehearsing the scenes, etc. An ounce of prevention is worth a pound of cure, you know it. 

If you have completely unfixable seams, you may have to reshoot some scenes. If there is no time or budget for a reshoot, then unfortunately the scene will have to be cut. If you have the budget for post, the shots can potentially be saved by applying some post production magic. 

If you are in an uncontrolled environment such as a live event, there may be 20-30 people walking in and out of seams. Stitch the static background first for these shots with multiple subjects in multiple seams. Then optimize the stitch on the main subjects in motion. Once there is the least amount of scenes possible, start brainstorming a creative solution. 


\imgA{1}{38/before}

\imgA{1}{38/after}

Try \textbf{\nameref{chap:50}} or \textbf{\nameref{chap:51}} techniques to fix the seams. If possible, have the shot cut altogether or look into alternatives such as rewriting the script or selecting a different shot that enhances the story.

\clearpage
\end{fullwidth}